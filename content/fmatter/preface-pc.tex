\section{Message from the Program Committee Co-Chairs}
\setheaders%
    {Message from the Program Committee Co-Chairs}%
    {Message from the Program Committee Co-Chairs}
\thispagestyle{emptyheader}
%\renewcommand{\large}{\fontsize{9}{11}\selectfont}
% that's a hack to make this part nicely fill the pages

\setlength{\parskip}{.7ex}
%\setlength{\parindent}{0pt}

Welcome to San Diego for the 15th Annual Conference of the North American 
Chapter of the Association for Computational Linguistics: Human Language 
Technologies!

The conference has grown remarkably in the past five years: we had 698 
submissions this year, despite our deadline right after the end-of-the-year 
holidays. As we worked on organizing the conference program, we made many 
changes to reflect the growth of the NAACL community, the increasing 
diversity of topics covered by the field, and the acceleration of the pace 
of the publication cycle.

We had a record short time between paper submission and author 
notification---less than two months. We settled on such compressed timeline 
in order to avoid spreading the reviewing period over the winter holidays, 
to ensure that papers spend only a short time under submission, and to 
coordinate submission deadlines with ACL. Our incredible team of area 
chairs and reviewers ensured that the planned schedule went smoothly.

As the computational linguistics field has expanded, it has become 
increasingly difficult to recruit a sufficient number of knowledgeable 
reviewers. We decided to reach out to the largest possible pool of 
computational linguists and provide convenient ways for the area chairs to 
control which reviewers they end up working with: we invited all 
researchers actively working in the area of computational 
linguistics/language processing to review for the conference. We defined 
``active researchers'' to be those who have published at least five papers 
in the last ten years in the ACL, NAACL, EMNLP, EACL or COLING conferences. 
In order to be inclusive of the amazing young researchers who became active 
in the field only more recently, we also included everyone who had published
at least three papers in the same venues for the last five years. This 
yielded a list of over 1,400 researchers that we invited to serve as 
reviewers for the conference. Of these, 685 agreed and participated in the 
review process. This is another record for NAACL HLT 2016, no previous 
NAACL has had such a large program committee. Among these, the area chairs 
recognized 120 as best reviewers.

Working with the reviewers were the 42 area chairs. We asked the area chairs
to work in pairs, so they can have a back-up in case other obligations need 
their attention during the review period and to ensure that all decisions 
about reviewer assignment and paper recommendation are discussed in detail. 
All area chairs and reviewers submitted a list of keywords that describe 
their area of expertise (the full list appears in the conference call for 
papers). The area chairs were paired based on the keyword overlap.

To match reviewers to area chairs, we used a bidding system. For bidding, 
each area received a list of the 140 reviewers with best matching keyword 
profiles. If the area chairs did not know the work of a potential reviewer 
on their bidding list, they looked him or her up on DBLP or Google Scholar 
before making their final bid. Areas were assigned only reviewers for which 
the area chairs bid positively. Area chairs were free as usual to recruit 
additional reviewers they wished to work with.

Submissions were assigned to areas by taking into account the match between 
the paper keywords and the area chair keywords. Areas were capped at 40 
submissions maximum (long and short combined). As in the past, reviewers 
bid on papers they wanted to review. 69\% of the reviews were written by 
reviewers who had bid indicating that they want to review the paper; 29\% 
of the reviews were written by reviewers who had bid indicating they are ok 
with reviewing the paper. The remaining 2\% of reviews were written by 
reviewers who did not bid on the paper but were asked by an area chair to 
review it. Three reviewers were assigned a paper that they did not want to 
review according to their bid. The average reviewer load was 3 papers, which
included a mix of long and short submissions. Only 43 reviewers had more 
than four papers to review.

Area chairs wrote meta-reviews, for use only by us, justifying their 
accept/reject recommendation. In making difficult decisions, we drew on 
these meta-reviews, the reviews themselves, the discussion among the 
reviewers, and the author response to the initial reviews.

We are happy with our changes to the review process: area chairs had control
over the reviewers they worked with, reviewers were assigned papers they 
wanted to review and the overall reviewing load was low. Needless to say, 
there is room for further improvements. The reviewing process is crucial to 
the quality of this conference; only if the community has confidence in the 
quality of the reviewing process will this conference continue to be a 
leading conference in our field. Our goal has been to make sure that every 
single submission receives a complete and fair review and decision, and to 
make sure that the authors of every single submission understand why their 
paper was accepted or declined for the conference. We would like to thank 
our 685 reviewers, and we would especially like to thank our 42 area chairs,
who were patient in allowing us to pursue some of the innovative aspects of 
this year’s reviewing cycle.

Eighteen of the 698 initial submissions were withdrawn by the authors or 
rejected without review because of formatting violations. A total of 396 
long and 284 short papers underwent review; 100 long and 82 short papers 
were accepted, for an acceptance rate of 25\% and 29\% respectively. In 
addition, ten TACL papers will be presented at the conference.

This year we decided to have shorter slots for oral presentations, in order 
to have more of the accepted papers presented as talks. In the program, 
long papers are allotted 20-minute slots (15 min presentation + 5 min 
questions). Short papers are allotted 10-minute slots (6 min presentation + 
4 min questions).

The best paper award committee consisted of NAACL general and program 
chairs from the last three years. Not all past chairs could participate in 
the selection. The final best paper committee included Joyce Chai, Katrin 
Kirchhoff, Rada Mihalcea, Kristina Toutanova, Lucy Vanderwende and Hua Wu. 
They selected two best long papers and one best short paper, along with two 
runner-ups in each category.

\noindent \textbf{Best Short Paper}\\
\emph{Improving sentence compression by learning to predict gaze}\\
Sigrid Klerke, Yoav Goldberg and Anders S{\o}gaard

\noindent \textbf{Short Paper, Runners Up}\\
\emph{Patterns of Wisdom: Discourse-Level Style in Multi-Sentence Quotations}\\
Kyle Booten and Marti A. Hearst

\noindent \emph{A Joint Model of Orthography and Morphological Segmentation}\\
Ryan Cotterell, Tim Vieira and Hinrich Schütze

\noindent \textbf{Best Long Papers}\\
\emph{Feuding Families and Former Friends: Unsupervised Learning for Dynamic Fictional Relationships}\\
Mohit Iyyer, Anupam Guha, Snigdha Chaturvedi, Jordan Boyd-Graber and Hal Daum\'{e} III

\noindent \emph{Learning to Compose Neural Networks for Question Answering}\\
Jacob Andreas, Marcus Rohrbach, Trevor Darrell and Dan Klein

\noindent \textbf{Long Paper, Runners Up}\\
\emph{Multi-way, Multilingual Neural Machine Translation with a Shared Attention Mechanism}\\
Orhan Firat, Kyunghyun Cho and Yoshua Bengio

\noindent \emph{Black Holes and White Rabbits: Metaphor Identification with Visual Features}\\
Ekaterina Shutova, Douwe Kiela and Jean Maillard

The conference program includes two inspiring invited talks by Regina 
Barzilay and Ehud Reiter. Both push the boundaries of the field, discussing 
the potential for real-world impact of language technologies. 

Finally we would like to thank all other people who supported us in the past
year in our work for NAACL HLT 2016. Last year’s program chairs, Anoop 
Sarkar and Joyce Chai shared their valuable advice and promptly answered the
many questions we had throughout the process. The NAACL board chair for 2015
(Hal Daum\'{e} III) and 2016 (Emily Bender) were our effective link with the
NAACL board. The conference general chair, Kevin Knight, was always 
available to us when we needed to consult about decisions we were making. 
The conference business manager, Priscilla Rasmussen, gave us details about 
the venue and coordinated with us at the final stages of making the 
conference schedule. The ACL treasurer, Greame Hirst, answered questions 
about the venue. The conference webmaster, Jason Riesa, put content on the 
conference webpage as soon as we made it available to him. The publication 
chairs, Meg Mitchell and Adam Lopez, answered all lingering author 
questions about formatting for submission and final versions. Many talks to 
all of them!

We look forward to an exciting conference!

\vskip 0.3in
\noindent NAACL HLT 2016 Program Co-Chairs \\
Ani Nenkova, University of Pennsylvania\\
Owen Rambow, Columbia University

