\documentclass[11pt]{article}
\usepackage[utf8]{inputenc} 
\usepackage[T1]{fontenc} % fonts to encode unicode
\usepackage{times}
\sloppy
\hyphenpenalty 10000

% for Letter size

\setlength\topmargin{0.2cm} \setlength\oddsidemargin{-0cm}
\setlength\textheight{22cm} \setlength\textwidth{15.8cm}
\setlength\columnsep{0.25in}  \newlength\titlebox \setlength\titlebox{2.00in}
\setlength\headheight{5pt}   \setlength\headsep{0pt}
\setlength\footskip{1.0cm}
\setlength\leftmargin{0.0in}
\pagestyle{empty}
%%%%%%%%%%%%%%%%%%%%%%%%%%%%%%%%%%%%%%%%%%%%%%%%%%%%%%%%%%%%%%%%


% for A4 size

%\setlength\topmargin{-5mm} \setlength\oddsidemargin{-0cm}
%\setlength\textheight{24.7cm} \setlength\textwidth{16cm}
%\setlength\columnsep{0.6cm}  \newlength\titlebox \setlength\titlebox{2.00in}
%\setlength\headheight{5pt}   \setlength\headsep{0pt}
%\setlength\footskip{1.0cm}
%\setlength\leftmargin{0.0in}
%\pagestyle{empty}
%%%%%%%%%%%%%%%%%%%%%%%%%%%%%%%%%%%%%%%%%%%%%%%%%%%%%%%%%%%%%%%%


\setlength{\parindent}{0in}
\setlength{\parskip}{2ex}

\begin{document}

\begin{center}
  {\Large \bf Introduction}
\end{center}

\vspace*{0.5cm}

%%%%%%%%%%%%%%%%%%%%%%%%%%%%%%%%%%%%%%%%%%%%%%%%%%%%%%%%%%%%%%%%%%%%%%%%

%%% INSERT YOUR INTRO HERE
Many NLP researchers, especially those not working in the area of discourse processing, tend to equate coreference resolution with the sort of coreference that people did in MUC, ACE, and OntoNotes, having the impression that coreference is a well-worn task owing in part to the large number of papers reporting results on the MUC/ACE/OntoNotes corpora. This is an unfortunate misconception: the previous SemEval 2010 and CoNLL 2012 shared tasks on coreference resolution have largely focused on entity coreference, which constitutes only one of the many kinds of coreference relations that were discussed in theoretical and computational linguistics in the past few decades. In fact, by focusing on entity coreference resolution, NLP researchers have only scratched the surface of the wealth of interesting problems in coreference resolution.

The decision to focus on entity coreference resolution was initially made by information extraction (IE) researchers when coreference was selected as one of the tasks in the MUC-6 coreference in 1995. Many interesting kinds of coreference relations, such as bridging and reference to abstract entities, were left out not because they were not important, but because “it was felt that the menu was simply too ambitious”. It turned out that this decision had an important consequence: the progress made in coreference research in the past two decades was largely driven by the availability of coreference-annotated corpora such as MUC, ACE, and OntoNotes, where entity coreference was the focus. 

Given the plethora of work on entity coreference and aware of other fora gathering coreference-related papers (such as LAW, DiscoMT or EVENTS), we believed that time was ripe for a new workshop on the single topic of coreference resolution that would bring together researchers who were interested in under-investigated coreference phenomena, willing to contribute both theoretical and applied computational work on coreference resolution, especially for languages other than English, less-researched forms of coreference and new applications of coreference resolution. 

Our call attracted 20 submissions out of which we have selected 4 long and 2 short papers for oral presentation and 7 papers for poster presentation based on reviewers' recommendations. 

We would like to thank the Program Committee members who reviewed the submissions and the workshop participants for joining in!

\begin{flushright}
--- Maciej Ogrodniczuk and Vincent Ng
\end{flushright}

\end{document}
