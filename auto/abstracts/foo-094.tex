Persuasive communication is the process of shaping, reinforcing and changing others' responses. In political debates, speakers express their views towards the debated topics through not only the choice of discourse content but also an argumentation process. In this work we study the use of semantic frames for modelling argumentation in speakers' discourse. We investigate the impact of a speaker's argumentation style and their effect in influencing an audience in supporting their candidature. We model the influence index of each candidate based on results of national polls occurring immediately after a debate took place and present a system which ranks speakers in terms of their relative influence based on a combination of content and persuasive argumentation features. Our results show that although content alone is predictive of a speaker's influence rank, persuasive argumentation also affects such indices.
