Alzheimer's disease (AD) and depression share a number of symptoms, and commonly occur together. Being able to differentiate between these two conditions is critical, as depression is generally treatable. We use linguistic analysis and machine learning to determine whether automated screening algorithms for AD are affected by depression, and to detect when individuals diagnosed with AD are also showing signs of depression. In the first case, we find that our automated AD screening procedure does not show false positives for individuals who have depression but are otherwise healthy. In the second case, we have moderate success in detecting signs of depression in AD (accuracy = 0.649), but we are not able to draw a strong conclusion about the features which are most informative to the classification.
