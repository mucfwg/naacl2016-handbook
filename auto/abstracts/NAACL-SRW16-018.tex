Sindhi is an Indo-Aryan language spoken by more than 75 million native speakers in the world. Nevertheless, Sindhi is a resource-poor language in terms of the availability of language technology tools and resources. In this thesis, we discuss the challenges faced and steps taken for creating raw and annotated datasets, constructing NLP tools such as a POS tagger, a morphological analyser, creating a transliteration system without parallel data in an unsupervised fashion and developing a SMT system for Sindhi-Urdu and adopting techniques for improving it.
