Interactive machine translation (IMT) is a method which uses human-computer interactions to improve the quality of MT. Traditional IMT methods employ a left-to-right order for the interactions, which is difficult to directly modify critical errors at the end of the sentence. In this paper, we propose an IMT framework in which the interaction is decomposed into two simple human actions: picking a critical translation error (Pick) and revising the translation (Revise). The picked phrase could be at any position of the sentence, which improves the efficiency of human computer interaction. We also propose automatic suggestion models for the two actions to further reduce the cost of human interaction. Experiment results demonstrate that by interactions through either one of the actions, the translation quality could be significantly improved. Greater gains could be achieved by iteratively performing both actions.
