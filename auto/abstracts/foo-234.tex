Computational approaches to simultaneous interpretation are held back by how little we know about the tactics human interpreters use. We produce a parallel corpus of translated and simultaneously interpreted text and study differences between them through a computational approach. Our analysis reveals that human interpreters regularly apply several effective tactics to reduce translation latency, including sentence segmentation and passivization. In addition to these unique, clever strategies, we show that limited human memory also causes other idiosyncratic properties of human interpretation such as generalization and omission of source content.
