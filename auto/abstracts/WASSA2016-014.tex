Negators, modals, and degree adverbs can significantly affect the sentiment of the words they modify. Often, their impact is modeled with simple heuristics; although, recent work has shown that such heuristics do not capture the true sentiment of multi-word phrases. We created a dataset of phrases that include various negators, modals, and degree adverbs, as well as their combinations. Both the phrases and their constituent content words were annotated with real-valued scores of sentiment association. Using phrasal terms in the created dataset, we analyze the impact of individual modifiers and the average effect of the groups of modifiers on overall sentiment. We find that the effect of modifiers varies substantially among the members of the same group. Furthermore, each individual modifier can affect sentiment words in different ways. Therefore, solutions based on statistical learning seem more promising than fixed hand-crafted rules on the task of automatic sentiment prediction.
