A simile is a figure of speech comparing two fundamentally different things. Sometimes, a simile will explain the basis of a comparison by explicitly mentioning a shared property. For example, ``my room is as cold as Antarctica'' gives ``cold'' as the property shared by the room and Antarctica. But most similes do not give an explicit property (e.g., ``my room feels like Antarctica'') leaving the reader to infer that the room is cold. We tackle the problem of automatically inferring implicit properties evoked by similes. Our approach involves three steps: (1) generating candidate properties from different sources, (2) evaluating properties based on the influence of multiple simile components, and (3) aggregated ranking of the properties. We also present an analysis showing that the difficulty of inferring an implicit property for a simile correlates with its interpretive diversity.
