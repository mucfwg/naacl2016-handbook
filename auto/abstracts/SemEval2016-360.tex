This paper describes the participation of the team ``TwiSE'' in the SemEval 2016 challenge. Specifically, we participated in Task 4, namely ``Sentiment Analysis in Twitter'' for which we implemented sentiment classification systems for subtasks A, B, C and D. Our approach consists of two steps. In the first step, we generate and validate diverse  feature sets for twitter sentiment evaluation, inspired by the work of participants of previous editions of such challenges. In the second step, we focus on the optimization of the evaluation measures of the different subtasks. To this end, we examine different learning strategies by validating them on the data provided by the task organisers. For our final submissions we used an ensemble learning approach (stacked generalization) for Subtask A and single linear models for the rest of the subtasks. In the official leaderboard we were ranked 9/35, 8/19, 1/11 and 2/14 for subtasks A, B, C and D respectively. The code can be found at \url{https://github.com/balikasg/SemEval2016-Twitter\_Sentiment\_Evaluation}.
