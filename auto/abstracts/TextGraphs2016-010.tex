Language processing tools suffer from significant performance drops in social media domain due to its continuously evolving language. Transforming non-standard words into their standard forms has been studied as a step towards proper processing of ill-formed texts. This work describes a normalization system that considers contextual and lexical similarities between standard and non-standard words for removing noise in texts. A bipartite graph that represents contexts shared by words in a large unlabeled text corpus is utilized for exploring normalization candidates via random walks. Input context of a non-standard word in a given sentence is tailored in cases where a direct match to shared contexts is not possible. The performance of the system was evaluated on Turkish social media texts.
