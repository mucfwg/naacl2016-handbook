Creole languages do not fit into the traditional tree model of evolutionary history because multiple languages are involved in their formation. In this paper, we present several statistical models to explore the nature of creole genesis. After reviewing quantitative studies on creole genesis, we first tackle the question of whether creoles are typologically distinct from non-creoles. By formalizing this question as a binary classification problem, we demonstrate that a linear classifier fails to separate creoles from non-creoles although the two groups have substantially different distributions in the feature space. We then model a creole language as a mixture of source languages plus a special restructurer. We find a pervasive influence of the restructurer in creole genesis and some statistical universals in it, paving the way for more elaborate statistical models.
