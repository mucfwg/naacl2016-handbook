Motif analysis counts the number of small building blocks (the motifs) in a network and relates these statistical numbers to the inherent semantics of the network. In the realm of natural language processing, the networks are induced by texts. We demonstrate that motif analysis may help assess the quality of a document. More specifically, we consider the German Wikipedia and use the label ``featured'' as the (binary) quality criterion. The length (number of words) of an article is a comparatively good predictor for this label. We show that a well-designed combination of this criterion and motif statistics yields a significant improvement. We also found that a deeper look into the most relevant motifs may improve our understanding of quality.
