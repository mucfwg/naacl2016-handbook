Dementia is an increasing problem for an aging population, with a lack of available treatment options, as well as expensive patient care. Early detection is critical to eventually postpone symptoms and to prepare health care providers and families for managing a patient's needs. Identification of diagnostic markers may be possible with patients' clinical records. Text portions of clinical records are integrated into predictive models of dementia development in order to gain insights towards automated identification of patients who may benefit from providers' early assessment. Results support the potential power of linguistic records for predicting dementia status, both in the absence of, and in complement to, corresponding structured nonlinguistic data.
