Sentiment analysis has become a well-established task in Natural Language Processing. As such, a high variety of methods have been proposed to tackle it, for different types of texts, text levels, languages, domains and formality levels.  Although state-of-the-art systems have obtained promising results, a big challenge that still remains is to port the systems to the ``real world'' --- i.e. to implement systems that are running around the clock, dealing with information of heterogeneous nature, from different domains, written in different styles and diverse in formality levels. The present paper describes our efforts to implement such a system, using a variety of strategies to homogenize the input and comparing various approaches to tackle the task. Specifically, we are tackling the task using two different approaches: a) one that is unsupervised, based on dictionaries of sentiment-bearing words and heuristics to compute final polarity of the text considered; b) the second, supervised, trained on previously annotated data from different domains. For both approaches, the data is first normalized and the slang is replaced with its expanded version.
