In this paper we study several approaches to adapting a polarity lexicon to a specific domain. On the one hand, the domain adaptation using Term Frequency (TF) and on the other hand, the domain adaptation using pattern matching with a BootStrapping algorithm (BS). Both methods are corpus based and start with the same polarity lexicon, but the first one requires an annotated collection of documents while the second one only needs a corpus where it looks for linguistic patterns. The performance of both methods overcomes the baseline system using the general polarity lexicon iSOL. However, although the TF approach achieves very promising results, the BS strategy does not give as much improvement as we expected. For this reason we have combined both methods, in order to take advantage of the positive aspects of each one. With this new approach the results obtained are even better that those with the systems applied individually. Actually, we have achieved a significant improvement of 11.50\% (in terms of accuracy) in the polarity classification of the movie reviews with respect to the results achieved with the general purpose lexicon iSOL.
