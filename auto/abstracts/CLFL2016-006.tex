The chiasmus is a rhetorical figure involving the repetition of a pair of words in reverse order, as in ``all for one, one for all''. Previous work on detecting chiasmus in running text has only considered superficial features like words and punctuation. In this paper, we explore the use of syntactic features as a means to improve the quality of chiasmus detection. Our results show that taking syntactic structure into account may increase average precision from about 40 to 65\% on texts taken from European Parliament proceedings. To show the generality of the approach, we also evaluate it on literary text and observe a similar improvement and a slightly better overall result.
