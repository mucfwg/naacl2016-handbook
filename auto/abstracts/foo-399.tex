Paronomasic puns create humor through the relationship between a pun and its phonologically similar target. For example, in ``Don't take geologists for granite'' the word ``granite'' is a pun with the target ``granted''. The recovery of the target in the mind of the listener is essential to the success of the pun. This work introduces a new model for automatic target recovery and provides the first empirical test for this task. The model draws upon techniques for automatic speech recognition using weighted finite-state transducers, and leverages automatically learned phone edit probabilities that give insight into how people perceive sounds and into what makes a good pun.  The model is evaluated on a small corpus where it is able to automatically recover a large fraction of the pun targets.
