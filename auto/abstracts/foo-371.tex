Multitask learning has been proven a useful technique in a number of Natural Language Processing applications where data is scarce and naturally diverse. Examples include learning from data of different domains and learning from annotations on translation quality provided by multiple annotators. {\em Tasks} in these scenarios would be the domains or the annotators. When faced with limited data for each task, a framework for the learning of tasks in parallel while using a shared representation is clearly helpful: what is learned for a given task can be transferred to other tasks while the particularities of each task are still modelled independently. Focusing on machine translation quality estimation as application, in this paper we show that multitask learning is also useful in cases where data is abundant. Based on two large-scale datasets, we explore models with multiple annotators and multiple languages and show that state-of-the-art multitask learning algorithms lead to improved results in all settings.
