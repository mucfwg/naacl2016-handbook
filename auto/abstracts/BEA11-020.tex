As many as two-thirds of individuals with an Autism Spectrum Disorder (ASD) also have language impairments, which can range from mild limitations to complete non-verbal behavior. For such cases, there are several augmentative and alternative communication (AAC) devices available. These are computer-designed tools in order to help people with ASD to palliate or overcome such limitations, at least partially. Some of the most popular AAC devices are based on pictograms, so that a pictogram is the graphical representation of a simple concept and sentences are composed by concatenating a number of such pictograms. Usually, these tools have to manage a vocabulary made up of several hundred of pictograms/concepts, with no or very poor knowledge of the language at semantic and pragmatic level. In this paper we present Pictogrammar, an AAC system which takes advantage of SUpO and PictOntology. SUpO (Simple Upper Ontology) is a formal semantic ontology which is made up of detailed knowledge of facts of everyday life such as simple words, with special interest in linguistic issues. PictOntology is an ontology developed to manage sets of pictograms, linked to SUpO. Both ontologies make possible the development of tools which are able to take advantage of a formal semantic.
