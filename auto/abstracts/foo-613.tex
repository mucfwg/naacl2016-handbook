Given the limited success of medication in reversing the effects of Alzheimer's and other dementias, a lot of the neuroscience research has been focused on early detection, in order to slow the progress of the disease through different interventions. We propose a Natural Language Processing approach applied to descriptive writing to attempt to discriminate decline due to normal aging from decline due to predementia conditions. Within the context of a longitudinal study on Alzheimer's disease, we created a unique corpus of 201 descriptions of a control image written by subjects of the study. Our classifier, computing linguistic features, was able to discriminate normal from cognitively impaired patients to an accuracy of 86.1\% using lexical and semantic irregularities found in their writing. This is a promising result towards elucidating the existence of a general pattern in linguistic deterioration caused by dementia that might be detectable from a subject's written descriptive language.
