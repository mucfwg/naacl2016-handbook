A deeper understanding of what is going on in a given text is still one of the most interesting and challenging goals in NLP. Sentiment analysis has recently started to contribute to this area. We no longer just try to predict the polarity of whole product reviews but to distinguish various perspectives inherent to a text, namely, what the author is telling us, how he implicitly or explictely evaluates it and what his text tells us about the attitudes the entities in the text hold towards each other (or towards the mentioned objects, situations, or opinions of others). Other perspectives not yet taken by our systems include the common-sense perspective (what follows from the behaviour of an agent for his perception by the public) or the reader perspective: given that I have specified the pros and cons of my world view - which are the opponents and proponents of mine given the text at hand. Progress has been made in this direction. Sentiment inferences based on verb-specific polar lexicons and general inference rules have been proposed (see the work of Deng and Wiebe, for instance). Our preprocessing tools for the extraction of predicate argument structures or for semantic role labeling seem to be mature enough to support this kind of deep understanding reasonably well.
