This paper presents a quantitative and qualitative error analysis of Russian anaphora resolvers which participated in the RU-EVAL event. Its aim is to identify and characterize a set of challenging errors common to state-of-the-art systems dealing with Russian. We examined three types of pronouns: 3rd person pronouns, reflexive and relative pronouns. The investigation has shown that a high level of grammatical ambiguity, specific features  of reflexive pronouns, free word order and special cases of non-referential pronouns in Russian impact the quality of anaphora resolution systems. Error analysis reveals some specific features of anaphora resolution for morphologically rich and free word order languages with a lack of gold standard resources.
