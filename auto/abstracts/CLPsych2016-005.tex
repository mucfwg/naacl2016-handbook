The need to protect privacy poses unique challenges to behavioral research. For instance, researchers often can not use examples drawn directly from such data to explain or illustrate key findings. In this research, we use data-driven models to synthesize realistic-looking data, focusing on discourse produced by social-media participants announcing life-changing events. We comparatively explore the performance of distinct techniques for generating synthetic linguistic data across different linguistic units and topics. Our approach offers utility not only for reporting on qualitative behavioral research on such data, where directly quoting a participant's content can unintentionally reveal sensitive information about the participant, but also for clinical computational system developers, for whom access to realistic synthetic data may be sufficient for the software development process. Accordingly, thi work also has implications for computational linguistics at large.
