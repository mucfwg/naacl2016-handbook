Clinical research article summaries called infoPOEMs (Patient-Oriented Evidence that Matters) are emailed by the Canadian Medical Association to family physicians who read them and answer the online Information Assessment Method (IAM) questionnaire which a free form textual opinion fields to comment on the value or content of the infoPOEM. This article presents results of a relevance evaluation study applied on these comments to automatically determine their helpfulness and consequently the interest of sharing them among the medical community. A dataset of 3,470 manually annotated comments provides a gold standard, containing structural, syntactic, and semantic features taken from the Unified Medical Language System and IAM questionnaire. Applied machine learning algorithms show a global f-measure improvement of 9.1\% when compared to a binary occurrence bag-of-word baseline.
