This paper focuses on the interaction of chains of coreference identity with other types of relations, comparing English and German data sets in terms of language, mode (written vs. spoken) and register. We first describe the types of coreference and the chain features analysed as indicators of textual coherence and topic continuity. After sketching the feature categories under analysis and the methods used for statistical evaluation, we present the findings from our analysis and interpret them in terms of the contrasts mentioned above. We will also show that for some registers, coreference types other than identity are of great importance.
