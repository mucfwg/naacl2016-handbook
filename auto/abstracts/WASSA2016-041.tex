Automatic affect detection and classification from text is a complex task in Natural Language Pro-cessing, whose tackling requires not only the use of established methods in the field, but also the use of knowledge extracted from theories in Psy-chology, Cognitive Science, Social Psychology or Neuropsychology. For the past decade, there has been a large amount of research done in the field. Neverthe-less, many issues remain to be tackled, starting from a common understanding of what we mean by the concepts involved and integrating the re-search in the appropriate context. In this intro talk, my aim is to give a broad over-view of the issues involved in tackling the task, from the definition of the terms, to some of the tasks that have been defined and some of the methods employed.
