Emotions are not linguistic entities but they are conveniently expressed through the language. Feelings influence actions, thoughts and of course our way of communicate. It was more than twelve years ago that we developed WordNet-Affect (Strapparava and Valitutti, 2004), and I remember that at that time several people questioned about the utility and even the possibility of studying emotions using computational linguistics techniques. In the recent years, nonetheless there has been a flourishing interest in automatically detecting and generating emotions in texts, with many valuable research contributions by the community. The space here is too short to think to even shortly mention and review them. Anyway I would like to indicate which directions are more promising in my opinion.
