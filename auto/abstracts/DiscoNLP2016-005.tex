Non-adjacent linguistic phenomena such as non-contiguous multiwords and other phrasal units containing insertions, i.e., words that are not part of the unit, are difficult to process and remain a problem for NLP applications. Non-contiguous multiword units are common across languages and constitute some of the most important challenges to high quality machine translation. This paper presents an empirical analysis of non-contiguous multiwords, and highlights our use of the Logos Model and the Semtab function to deploy semantic knowledge to align non-contiguous multiword units with the goal to translate these units with high fidelity. The phrase level manual alignments illustrated in the paper were produced with the CLUE-Aligner, a Cross-Language Unit Elicitation alignment tool.
