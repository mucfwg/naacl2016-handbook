Mental Health Records (MHRs) contain free-text documentation about patients' suicide and suicidality. In this paper, we address the problem of determining whether grammatic variants (inflections) of the word ``suicide'' are affirmed or negated. To achieve this, we populate and annotate a dataset with over 6,000 sentences originating from a large repository of MHRs. The resulting dataset has high Inter-Annotator Agreement (kappa 0.93). Furthermore, we develop and propose a negation detection method that leverages syntactic features of text[https://github.com/gkotsis/negation-detection]. Using parse trees, we build a set of basic rules that rely on minimum domain knowledge and render the problem as binary classification (affirmed vs. negated). Since the overall goal is to identify patients who are expected to be at high risk of suicide, we focus on the evaluation of positive (affirmed) cases as determined by our classifier. Our negation detection approach yields a recall (sensitivity) value of 94.6\% for the positive cases and an overall accuracy value of 91.9\%. We believe that our approach can be integrated with other clinical Natural Language Processing tools in order to further advance information extraction capabilities.
