People often use social media to discuss opinions, including political ones. We refer to relevant topics in these discussions as political issues, and the alternate stands towards these topics as political positions. We present a Political Issue Extraction (PIE) model that is capable of discovering political issues and positions from an unlabeled dataset of tweets. A strength of this model is that it uses twitter timelines of political and non-political authors, and affiliation information of only political authors. The model estimates word-specific distributions (that denote political issues and positions) and hierarchical author/group-specific distributions (that show how these issues divide people). Our experiments using a dataset of 2.4 million tweets from the US show that this model effectively captures the desired properties (with respect to words and groups) of political discussions. We also evaluate the two components of the model by experimenting with: (a) Use to alternate strategies to classify words, and (b) Value addition due to incorporation of group membership information. Estimated distributions are then used to predict political affiliation with 68\% accuracy.
