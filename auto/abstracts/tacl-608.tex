We consider the problem of disambiguating concept mentions appearing in documents and grounding them in multiple knowledge bases, where each knowledge base addresses some aspects of the domain. This problem poses a few additional challenges beyond those addressed in the popular Wikification problem. Key among them is that most knowledge bases do not contain the rich textual and structural information Wikipedia does; consequently, the main supervision signal used to train Wikification rankers does not exist anymore. In this work we develop an algorithmic approach that, by carefully examining the relations between various related knowledge bases, generates an indirect supervision signal it uses to train a ranking model that accurately chooses knowledge base entries for a given mention; moreover, it also induces prior knowledge that can be used to support a global coherent mapping of all the concepts in a given document to the knowledge bases. Using the biomedical domain as our application, we show that our indirectly supervised ranking model outperforms other unsupervised baselines and that the quality of this indirect supervision scheme is very close to a supervised model. We also show that considering multiple knowledge bases together has an advantage over grounding concepts to each knowledge base individually.