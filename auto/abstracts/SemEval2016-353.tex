This paper describes the system developed for the task of temporal information extraction from clinical narratives in the context of 2016 Clinical TempEval challenge. Clinical TempEval 2016 addressed the problem of temporal reasoning in clinical domain by providing annotated clinical notes and pathology reports similar to Clinical TempEval challenge 2015. The Clinical TempEval challenge consisted of six subtasks. Hitachi team participated in two time expression based subtasks: time expression span detection (TS) and time expression attribute identification (TA) for which we developed hybrid of rule-based and machine learning based methods using Stan-ford TokensRegex framework and Stanford Named Entity Recognizer and evaluated it on the THYME corpus. Our hybrid system achieved a maximum F-score of 0.73 for identification of time spans (TS) and 0.71 for identification of time attributes (TA).
