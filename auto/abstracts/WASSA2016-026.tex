This work explores the relationship between the sentiment of lyrics in Billboard Top 100 songs, stocks, and a consumer confidence index. We hypothesized that sentiment of Top 100 songs could be representative of public mood and correlate to stock market changes as well. We analyzed the sentiment for polarity and mood in terms of seven dimensions. We gathered data from 2008 to 2013 and found statistically significant correlations between lyrical sentiment polarity and DJIA closing values and between anxiety in lyrics and consumer confidence. We also found strong Granger-causal relationships involving anxiety, hope, anger, and both societal indicators. Finally, we introduced a vector autoregression model with time lag which is able to capture stock and consumer confidence indices (R-squared=.97, p<.001 and R-squared=0.72, p<.01 respectively).
