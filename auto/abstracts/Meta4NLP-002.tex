The elasticity of metaphor as a communication strategy has spurred philosophers to question its ability to mean anything at all. If a metaphor can elicit different responses from different people in varying contexts, how can one say it has a single meaning? Davidson has argued that metaphors have no special or secondary meaning, and must thus mean exactly what they seem to mean on the surface. It is this literally anomalous meaning that directs us to the pragmatic inferences that a speaker actually wishes us to explore. Conveniently, this laissez faire strategy assumes that metaphors are crafted from apt knowledge by speakers with real communicative intent, allowing useful inference to be extracted from their words. But this assumption is not valid in the case of many machine-generated metaphors that merely echo the linguistic form --- but lack the actual substance --- of real metaphors. We present here an open public resource with which a metaphor-generation system can give its figurative efforts real meaning.
