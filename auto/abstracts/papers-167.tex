We provide the first extensive evaluation of how using different types of context to learn skip-gram word embeddings affects performance on a wide range of intrinsic and extrinsic NLP tasks. Our results suggest that while intrinsic tasks tend to exhibit a clear preference to particular types of contexts and higher dimensionality, more careful tuning is required for finding the optimal settings for most of the extrinsic tasks that we considered. Furthermore, for these extrinsic tasks, we find that once the benefit from increasing the embedding dimensionality is mostly exhausted, simple concatenation of word embeddings, learned with different context types, can yield further performance gains. As an additional contribution, we propose a new variant of the skip-gram model that learns word embeddings from weighted contexts of substitute words.
