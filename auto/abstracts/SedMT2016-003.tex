This paper describes a deterministic method for generating natural language suited to being part of a machine translation system with meaning representations as the level for language transfer. Starting from Davidsonian/Penman meaning representations, syntactic trees are built following the Penn Parsed Corpus of Modern British English, from which the yield (i.e., the words) can be taken. The novel contribution is to highlight exploiting the presentation of meaning content to inform decisions regarding the selection of language constructions: active vs. passive, argument subject vs. expletive it vs. existential there, discourse vs. intra-sentential coordination vs. adverbial clause vs. participial clause vs. purpose clause, and infinitive clause vs. finite clause vs. small clause vs. relative clause vs. it cleft.
