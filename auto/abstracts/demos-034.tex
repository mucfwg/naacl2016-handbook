Foreign students at German universities often have difficulties following lectures as they are often held in German. Since human interpreters are too expensive for universities we are addressing this problem via speech translation technology deployed in KIT's lecture halls. Our simultaneous lecture translation system automatically translates lectures from German to English in real-time. Other supported language directions are English to Spanish, English to French, English to German and German to French. Automatic simultaneous translation is more than just the concatenation of automatic speech recognition and machine translation technology, as the input is an unsegmented, practically infinite stream of spontaneous speech. The lack of segmentation and the spontaneous nature of the speech makes it especially difficult to recognize and translate it with sufficient quality. In addition to quality, speed and latency are of the utmost importance in order for the system to enable students to follow lectures. In this paper we present our system that performs the task of simultaneous speech translation of university lectures by performing speech translation on a stream of audio in real-time and with low latency. The system features several techniques beyond the basic speech translation task, that make it fit for real-world use. Examples of these features are a continuous stream speech recognition without any prior segmentation of the input audio, punctuation prediction, run-on decoding and run-on translation with continuously updating displays in order to keep the latency as low as possible.
