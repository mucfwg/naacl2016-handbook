The sharing of emotional material is central to the process of psychotherapy and emotional problems are a primary reason for seeking treatment. Surprisingly, very little systematic research has been done on patterns of emotional exchange during psychotherapy.  It is likely that a major reason for this void in the research is the enormous cost of annotating sessions for affective content.  In the field of NLP, there have been major strides in the creation of algorithms for sentiment analysis, but most of this work has focused on written reviews of movies and twitter feeds with little work on spoken dialogue.  We have created a new database of 97,497 utterances from psychotherapy transcripts labeled by humans for sentiment.  We describe this dataset and present initial results for models identifying sentiment. We also show that one of the best models from the literature, trained on movie reviews, performed below many of our baseline models that trained on the psychotherapy corpus.
