We describe results of investigation of a specific type of discontinuous constructions, namely non-projective constructions concerning verbs and their arguments. This topic is especially important for languages with a relatively free word order, such as Czech, which is the language we have primarily worked with. For comparison, we have included some results for English. The corpora used for both languages are the Prague Czech-English Dependency Treebank and the Prague Dependency Treebank, which are both annotated at a dependency syntax level as well as a deep (semantic) level, including verbs and their valency (arguments). We are using traditionally defined non-projectivity on trees with full linear ordering, but the two levels of annotation are innovatively combined to determine if a particular (deep) verb-argument structure is non-projective. As a result, we have identified several types of discontinuities, which we classify either by the verb class or structurally in terms of the verb, its arguments and their dependents. In addition, we have quantitatively compared selected phenomena found in Czech translated texts (in the PCEDT) to the native Czech as found in the original Prague Dependency Treebank.
