A major benefit of tree-to-tree over tree-to-string translation is that we can use target-side syntax to improve reordering. While this is relatively simple for binarized constituency parses, the reordering problem is considerably harder for dependency parses, in which parent nodes can have arbitrarily many children. Previous approaches have tackled this problem by restricting grammar rules, reducing the expressive power of the translation model. In this paper we propose a general model for dependency tree-to-tree reordering based on flexible non-terminals that can compactly encode multiple insertion positions. We explore how insertion positions can be selected even in cases where rules do not entirely cover the children of input sentence words. The proposed method greatly improves the flexibility of translation rules at the cost of only a 30\% increase in decoding time, and we demonstrate a 1.2-1.9 BLEU improvement over a strong tree-to-tree baseline.
