Representation and learning of commonsense knowledge is one of the foundational problems in the quest to enable deep language understanding. This issue is particularly challenging for understanding casual and correlational relationships between events. While this topic has received a lot of interest in the NLP community, research has been hindered by the lack of a proper evaluation framework. This paper attempts to address this problem with a new framework for evaluating story understanding and script learning: the 'Story Cloze Test'. This test requires a system to choose the correct ending to a four-sentence story. We created a new corpus of {\textasciitilde}50k five-sentence commonsense stories, ROC-Short-Stories, to enable this evaluation. This corpus is unique in two ways: (1) it captures a rich set of causal and temporal commonsense relations between daily events, and (2) it is a high quality collection of everyday life stories that can also be used for story generation. Experimental evaluation shows that a host of baselines and state-of-the-art models based on shallow language understanding struggle to achieve a high score on the Story Cloze Test. We discuss these implications for script and story learning, and offer suggestions for deeper language understanding.
