Precise semantic representation of a sentence and definitive information extraction are key steps in the accurate processing of sentence meaning, especially for figurative phenomena such as sarcasm, Irony, and metaphor cause literal meanings to be discounted and secondary or extended meanings to be intentionally profiled. Semantic modelling faces a new challenge in social media, because grammatical inaccuracy is commonplace yet many previous state-of-the-art methods exploit grammatical structure. For sarcasm detection over social media content, researchers so far have counted on Bag-of-Words(BOW), N-grams etc. In this paper, we propose a neural network semantic model for the task of sarcasm detection. We also review semantic modelling using Support Vector Machine (SVM) that employs constituency parse-trees fed and labeled with syntactic and semantic information. The proposed neural network model composed of Convolution Neural Network(CNN) and followed by a Long short term memory (LSTM) network and finally a Deep neural network(DNN). The proposed model outperforms state-of-the-art text-based methods for sarcasm detection, yielding an F-score of .92.
