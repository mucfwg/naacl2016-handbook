Recent work on metaphor processing often employs selectional preference information. We present a comparison of different approaches to the modelling of selectional preferences, based on various ways of generalizing over corpus frequencies. We evaluate on the VU Amsterdam Metaphor corpus, a broad corpus of metaphor. We find that using only selectional preference information is not enough to  outperform an all-metaphor baseline classification. A possible explanation for this lies in the nature of the evaluation data, and lack of power of selectional preference information on its own for non-novel metaphor detection. To better investigate the role of metaphor type in metaphor detection, we suggest a resource with annotation of novel metaphor should be created.
