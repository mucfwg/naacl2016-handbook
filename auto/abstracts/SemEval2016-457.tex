In many areas, such as social science, politics or market research, people need to track sentiment and their changes over time. For sentiment analysis in this field it is more important to correctly estimate proportions of each sentiment expressed in the set of documents (quantification task) than to accurately estimate sentiment of a particular document (classification). Basically, our study was aimed to analyze the effectiveness of two iterative quantification techniques and to compare their effectiveness with baseline methods. All the techniques are evaluated using a set of synthesized data and the SemEval-2016 Task4 dataset. We made the quantification methods from this paper available as a Python open source library. The results of comparison and possible limitations of the quantification techniques are discussed.
