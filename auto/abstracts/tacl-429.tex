We define two proper subclasses of subsequential functions based on the concept of Strict Locality (McNaughton and Papert, 1971; Rogers and Pullum, 2011; Rogers et al., 2013) for formal languages. They are called Input and Output Strictly Local (ISL and OSL). We provide an automata-theoretic characterization of the ISL class and theorems establishing how the classes are related to each other and to Strictly Local languages. We give evidence that local phonological and morphological processes belong to these classes. Finally we provide a learning algorithm which provably identifies the class of ISL functions in the limit from positive data in polynomial time and data. We demonstrate this learning result on appropriately synthesized artificial corpora. We leave a similar learning result for OSL functions for future work and suggest future directions for addressing non-local phonological processes.