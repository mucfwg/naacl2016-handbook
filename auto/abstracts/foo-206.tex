Many languages use honorifics to express politeness, social distance, or the relative social status between the speaker and their addressee(s). In machine translation from a language without honorifics such as English, it is difficult to predict the appropriate honorific, but users may want to control the level of politeness in the output. In this paper, we perform a pilot study to control honorifics in neural machine translation (NMT) via side constraints, focusing on English-German. We show that by marking up the (English) source side of the training data with a feature that encodes the use of honorifics on the (German) target side, we can control the honorifics produced at test time. Experiments show that the choice of honorifics has a big impact on translation quality as measured by BLEU, and oracle experiments show that substantial improvements are possible by constraining the translation to the desired level of politeness.
