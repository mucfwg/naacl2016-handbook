Group discussions are essential for organizing every aspect of modern life, from faculty meetings to senate debates, from grant review panels to papal conclaves.  While costly in terms of time and organization effort, group discussions are commonly seen as a way of reaching better decisions compared to solutions that do not require coordination between the individuals (e.g. voting)---through discussion, the sum becomes greater than the parts.  However, this assumption is not irrefutable: anecdotal evidence of wasteful discussions abounds, and in our own experiments we find that over 30\% of discussions are unproductive. We propose a framework for analyzing conversational dynamics in order to determine whether a given task-oriented discussion is worth having or not.  We exploit conversational patterns reflecting the flow of ideas and the balance between the participants, as well as their linguistic choices. We apply this framework to conversations naturally occurring in an online collaborative world exploration game developed and deployed to support this research.  Using this setting, we show that linguistic cues and conversational patterns extracted from the first 20 seconds of a team discussion are predictive of whether it will be a wasteful or a productive one.
