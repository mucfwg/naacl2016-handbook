As more historical texts are digitized, there is interest in applying natural language processing tools to these archives. However, the performance of these tools is often unsatisfactory, due to language change and genre differences. Spelling normalization heuristics are the dominant solution for dealing with historical texts, but this approach fails to account for changes in usage and vocabulary. In this empirical paper, we assess the capability of domain adaptation techniques to cope with historical texts, focusing on the classic benchmark task of part-of-speech tagging. We evaluate several domain adaptation methods on the task of tagging Early Modern English and Modern British English texts in the Penn Corpora of Historical English. We demonstrate that the Feature Embedding method for unsupervised domain adaptation outperforms word embeddings and Brown clusters, showing the importance of embedding the entire feature space, rather than just individual words. Feature Embeddings also give better performance than spelling normalization, but the combination of the two methods is better still, yielding a 5\% raw improvement in tagging accuracy on Early Modern English texts.
