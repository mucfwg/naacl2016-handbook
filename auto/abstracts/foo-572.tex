Instantiation is a fairly common discourse relation and past work has suggested that it plays special roles in local coherence, in sentiment expression and in content selection in summarization. In this paper we provide the first systematic corpus analysis of the relation and show that relation-specific features can improve considerably the detection of the relation.  We show that sentences involved in Instantiation are set apart from other sentences by the use of gradable (subjective) adjectives, the occurrence of rare words and by different patterns in part of speech use. Words across arguments of Instantiation are connected through hypernym and meronym relations significantly more often than in other sentences and that they stand out in context by being significantly less similar to each other than other adjacent sentence pairs. These factors provide substantial predictive power that improves the identification of Instantiation by more than 5\% F-measure.
