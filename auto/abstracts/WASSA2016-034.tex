The recent rise of social media has greatly democratized content creation. Facebook, Twitter, Skype, Whatsapp and LiveJournal are now commonly used to share thoughts and opinions about anything in the surrounding world. This proliferation of social media content has created new opportunities to study public opinion, with Twitter being especially popular for research due to its scale, representativeness, variety of topics discussed, as well as ease of public access to its messages. Unfortunately, research in that direction was hindered by the unavailability of suitable datasets and lexicons for system training, development and testing. While some Twitter-specific resources were developed, initially they were either small and proprietary, such as the i-sieve corpus (Kouloumpis et al., 2011), were created only for Spanish like the TASS corpus (Villena-Rom´an et al., 2013), or relied on noisy labels obtained automatically (Mohammad, 2012; Pang et al., 2002). This situation changed with the shared task on Sentiment Analysis on Twitter, which was organized at SemEval, the International Workshop on Semantic Evaluation, a semantic evaluation forum previously known as SensEval. The task ran in 2013, 2014, 2015 and 2016, attracting over 40+ of participating teams in all four editions.
