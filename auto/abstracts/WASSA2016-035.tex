Among the many challenges that sentiment analysis (SA) faces, I want to concentrate on one which has not received much attention within the SA community, but that is going to play a major role in future applications: Sentiment Quantification (SQ) (Esuli and Sebastiani, 2010). Quantification is defined as the task of estimating the prevalence (i.e., relative frequency) of the classes of interest in a set of unlabelled data via supervised learning (Forman, 2008); examples of SQ are (i) determining the prevalence of endorsements in a set of tweets about a political candidate, or (ii) determining the prevalence of rebuttals in a set of reviews of a given book. A na¨ıve way to tackle quantification is by classifying each unlabelled item independently and computing the fraction of such items that have been attributed the class. However, a good classifier is not necessarily a good quantifier: assuming the binary case, even if (FP +FN) is comparatively small, bad quantification accuracy results if FP and FN are significantly different (since perfect quantification coincides with the case FP = FN). This has led researchers to study quantification as a task on its own right, rather than as a byproduct of classification.
