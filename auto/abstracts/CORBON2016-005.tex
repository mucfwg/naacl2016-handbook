In this paper, we present a syntactic approach to the annotation of bridging relations, so-called genitive bridging. We introduce the RuGenBridge corpus for Russian annotated with genitive bridging and compare it to the semantic approach that was applied in the Prague Dependency Treebank for Czech. We discuss some special aspects of bridging resolution for Russian and specifics of bridging annotation for languages where definite nominal groups are not as frequent as e.g. in Romance and Germanic languages. To verify the consistency of our method, we carry out two comparative experiments: the annotation of a small portion of our corpus with bridging relations according to both approaches and finding for all relations from the RuGenBridge their semantic interpretation that would be annotated for Czech.
