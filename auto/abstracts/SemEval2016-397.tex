We present an approach for tackling the tweet quantification problem in SemEval 2016. The approach is based on the creation of a cooccurrence graph per sentiment from the training dataset and a graph per topic from the test dataset with the aim of comparing each topic graph against the sentiment graphs and evaluate the similarity between them. A heuristic is applied on those similarities to calculate the percentage of positive and negative texts. The overall result obtained for the test dataset according to the proposed task score (KL divergence) is 0.261, showing that the graph based representation and heuristic could be a way of quantifying the percentage of tweets that are positive and negative in a given set of texts about a topic.
