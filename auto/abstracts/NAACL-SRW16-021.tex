Computational phylogenetics has been shown to be effective over grammatical characteristics. Recent work suggests that constraint-based formalisms are compatible with such an approach (Eden, 2013). In this paper, we report on simulations to determine how useful constraint-based formalisms are in phylogenetic research and under what conditions. Popular computational methods for phylogenetic research (estimating the evolutionary histories of languages) primarily involve comparisons over cognate sets (Nichols and Warnow, 2008).  Recent works (Dunn et al., 2005; Longobardi and Guardiano, 2009) indicate that comparing sets of grammatical parameters can be effective as well.  However, generating a large number of meaningful parameters remains a formal obstacle.  In this paper we argue that constraint-based grammar formalisms may be exploited for parameter generation, and explore to what extent such research is feasible.
