Moving from limited-domain natural language generation (NLG) to open domain is difficult because the number of semantic input combinations grows exponentially with the number of domains. Therefore, it is important to leverage existing resources and exploit similarities between domains to facilitate domain adaptation. In this paper, we propose a procedure to train multi-domain, Recurrent Neural Network-based (RNN) language generators via multiple adaptation steps. In this procedure, a model is first trained on counterfeited data synthesised from an out-of-domain dataset, and then fine tuned on a small set of in-domain utterances with a discriminative objective function. Corpus-based evaluation results show that the proposed procedure can achieve competitive performance in terms of  BLEU score and slot error rate while significantly reducing the data needed to train generators in new, unseen domains. In subjective testing, human judges confirm that the procedure greatly improves generator performance when only a small amount of data is available in the domain.
