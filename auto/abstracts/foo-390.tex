This paper presents an study of the use of interlocking phrases in phrase-based statistical machine translation.  We examine the effect on translation quality when the translation units used in the translation hypotheses are allowed to overlap on the source side, on the target side and on both sides. A large-scale evaluation on 380 language pairs was conducted. Our results show that overall the use of overlapping phrases improved translation quality by 0.3 BLEU points on average. Further analysis revealed that language pairs requiring a larger amount of re-ordering benefited the most from our approach. When the evaluation was restricted to such pairs, the average improvement increased to up to 0.75 BLEU points with over 97\% of the pairs improving. Our approach requires only a simple modification to the decoding algorithm and we believe it should be generally applicable to improve the performance of phrase-based decoders.
