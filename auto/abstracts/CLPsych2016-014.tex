Many people with mental illnesses face challenges posed by stigma perpetuated by fear and misconception in society at large. This societal stigma against mental health conditions is present in everyday language. In this study we take a set of 14 words with the potential to stigmatize mental health and sample Twitter as an approximation of contemporary discourse. Annotation reveals that these words are used with different senses, from expressive to stigmatizing to clinical. We use these word-sense annotations to extract a set of mental health--aware Twitter users, and compare their language use to that of an age- and gender-matched comparison set of users, discovering a difference in frequency of stigmatizing senses as well as a change in the target of pejorative senses. Such analysis may provide a first step towards a tool with the potential to help everyday people to increase awareness of their own stigmatizing language, and to measure the effectiveness of anti-stigma campaigns to change our discourse.
