This paper discusses genitive phrases in Hindi/Urdu in general and puts a particular focus on genitive scrambling, a process whereby the basic order of constituents is changed. In Hindi/Urdu, genitive phrases may not only occur at different structural positions within the NP that they modify; under the right circumstances, they can also be found outside of the NP, yielding discontinuous structures. The theoretical challenge is to identify and formalize the linguistic constraints that govern genitive scrambling. Further, a successful computational treatment correctly attaches the genitive phrase to its head NP. I use a Lexical-Functional Grammar to solve both challenges, demonstrating that the constraints can be aptly formulated using a functional uncertainty path. Successful attachment further depends on the morphological agreement of the genitive phrase with its head. On a theoretical level, the present contribution sheds light on the possibilities of NP discontinuities in a morphologically rich language like Hindi/Urdu.
