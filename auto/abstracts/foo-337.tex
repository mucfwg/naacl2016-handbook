An increasing amount of research has recently focused on representing affective states as continuous numerical values on multiple dimensions, such as the valence-arousal (VA) space. Compared to the categorical approach that represents affective states as several classes (e.g., positive and negative), the dimensional approach can provide more fine-grained sentiment analysis. However, affective resources with valence-arousal ratings are still very rare, especially for the Chinese language. Therefore, this study builds 1) an affective lexicon called Chinese valence-arousal words (CVAW) containing 1,653 words, and 2) an affective corpus called Chinese valence-arousal text (CVAT) containing 2,009 sentences extracted from web texts. To improve the annotation quality, a corpus cleanup procedure is used to remove outlier ratings and improper texts. Experiments using CVAW words to predict the VA ratings of the CVAT corpus show results comparable to those obtained using English affective resources.
